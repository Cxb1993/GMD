
\documentclass[11pt]{article}
\usepackage{times}
%\usepackage[a4paper, total={6in, 8in}]{geometry}
%\usepackage{fullpage} % Package to use full page
%\usepackage{parskip} % Package to tweak paragraph skipping
\setlength{\parskip}{3em}%
%\setlength{\parindent}{4em}%

\usepackage{enumitem}
\usepackage{graphicx}
\usepackage{subcaption}
\usepackage{float}
\usepackage[section]{placeins}
%\usepackage{tikz} % Package for drawing
\usepackage{amssymb}
\usepackage{amstext}
\usepackage{amsmath}
\usepackage{hyperref}
\usepackage[numbers]{natbib}
\setlength\parindent{0pt}
\setcounter{section}{0}
\renewcommand{\baselinestretch}{1.5}

\begin{document}

\begin{itemize}

\item {\it Please correct the missing references (see ?? in the citations in the text).}

Fixed -- sorry for the oversight.


\item {\it Also, please clarify briefly the statement "Access permission will be given upon request." What license governs the distribution and use of your code? Who will be given access, and who may be denied access? }

The Plume-SPH code, together with a user manual providing instructions for installation, running and visualization are archived at \url{ https://zenodo.org/record/572819#.WRCy7xiZORs} (DOI: 10.5281/zenodo.572819). The input data for all simulations presented in this work are archived in the same repository. The MPI license governs the distribution and use of the code and associated documentation files. Permission is granted, free of charge, to any person to deal in the software without restriction. 

Output of simulations presented in this paper are archived in Box. These are the ones for which ``Access permission will be given upon request."
Access will be provided to all requestors. Code will eventually migrate to the vhub.org platform also.

\item {\it More generally, please sharpen the abstract by clarifying what elements does this work bring that are important to other modelers (i.e., why should the paper be published in GMD and not elsewhere) - the software? new mathematical equations? novel computational techniques that other modelers can implement in their own code? etc.}

Modified abstract to highlight contributions of value to other modelers. Changes highlighted below.

\end{itemize}



{\bf  Plume-SPH provides the the first particle based simulation of volcanic plumes.  SPH (smoothed particle hydrodynamics) has several advantages over currently used mesh based methods in modeling of multiphase free boundary flows like volcanic plumes. }This tool will provide more accurate eruption source terms to users of VATDs (Volcanic ash transport and dispersion models) greatly improving volcanic ash forecasts.  The accuracy of these terms is crucial for forecasts from VATDs and the 3D SPH model presented here will provide better numerical accuracy. As an initial effort to exploit the feasibility and advantages of SPH in volcanic plume modeling, we adopt a relatively simple physics model  (3D dusty-gas dynamic model assuming well mixed eruption materiel and dynamic and thermodynamic equilibrium between air and erupted material and minimal effect of winds) targeted at capturing the salient features of a volcanic plume.  {\bf The documented open source code (MPI License)  is easily obtained and extended to incorporate other models of physics of interest to the large community of researchers investigating multiphase free boundary flows of volcanic or other origins. }

The Plume-SPH code also incorporates  several newly developed techniques in SPH  needed to address numerical challenges in simulating multiphase compressible turbulent flow.  {\bf The code should thus be also of general interest to the much larger community of researchers using and developing SPH based tools.} In particular, the $SPH-\varepsilon$ turbulence model is to capture mixing at unresolved scales, heat exchange due to turbulence is calculated by a Reynolds analogy and a corrected SPH is used to handle tensile instability and deficiency of particle distribution near the boundaries. We also developed methodology to impose velocity inlet and pressure outlet boundary conditions, both of which are scarce in traditional implementations of SPH. 

{\bf The core solver of our model is parallelized with MPI (message passing interface) obtaining good weak and strong scalability using novel techniques for data management using a SFCs (space-filling curves) and object creation time based indexing and hash table based storage scheme. These techniques are of  interest to  researchers engaged in developing particle in cell type methods. } The model is verified by comparing velocity and concentration distribution along the central axis and on the transverse cross with experimental results of JPUE (jet or plume that is ejected from a nozzle into a uniform environment) and the top height of the Pinatubo eruption of 15 June 1991. Our results are consistent with both observations and existing 3D plume models. Profiles of several integrated variables are compared with those calculated in existing 3D plume models, and further verify our model. Analysis of the plume evolution process illustrates that this model is able to reproduce the physics of plume development. 


\end{document}